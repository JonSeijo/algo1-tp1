\documentclass[a4paper]{article}

\usepackage[spanish]{babel} % Le indicamos a LaTeX que vamos a escribir en espa�ol.
\usepackage[latin1]{inputenc} % Permite utilizar tildes y e�es normalmente
\usepackage{ifthen}
\usepackage{amssymb}
\usepackage{multicol}
\usepackage{graphicx}
\usepackage[absolute]{textpos}
\makeatletter

\@ifclassloaded{beamer}{%
  \newcommand{\tocarEspacios}{%
    \addtolength{\leftskip}{4em}%
    \addtolength{\parindent}{-3em}%
  }%
}
{%
  \usepackage[top=1cm,bottom=2cm,left=1cm,right=1cm]{geometry}%
  \usepackage{color}%
  \newcommand{\tocarEspacios}{%
    \addtolength{\leftskip}{5em}%
    \addtolength{\parindent}{-3em}%
  }%
}

\newcommand{\encabezadoDeProblema}[4]{%
  % Ponemos la palabrita problema en tt
%  \noindent%
  {\normalfont\bfseries\ttfamily problema}%
  % Ponemos el nombre del problema
  \ %
  {\normalfont\ttfamily #2}%
  \ 
  % Ponemos los parametros
  (#3)%
  \ifthenelse{\equal{#4}{}}{}{%
  \ =\ %
  % Ponemos el nombre del resultado
  {\normalfont\ttfamily #1}%
  % Por ultimo, va el tipo del resultado
  \ : #4}
}

\newcommand{\encabezadoDeTipo}[2]{%
  % Ponemos la palabrita tipo en tt
  {\normalfont\bfseries\ttfamily tipo}%
  % Ponemos el nombre del tipo
  \ %
  {\normalfont\ttfamily #2}%
  \ifthenelse{\equal{#1}{}}{}{$\langle$#1$\rangle$}
}

% Primero definiciones de cosas al estilo title, author, date

\def\materia#1{\gdef\@materia{#1}}
\def\@materia{No especifi\'o la materia}
\def\lamateria{\@materia}

\def\cuatrimestre#1{\gdef\@cuatrimestre{#1}}
\def\@cuatrimestre{No especifi\'o el cuatrimestre}
\def\elcuatrimestre{\@cuatrimestre}

\def\anio#1{\gdef\@anio{#1}}
\def\@anio{No especifi\'o el anio}
\def\elanio{\@anio}

\def\fecha#1{\gdef\@fecha{#1}}
\def\@fecha{\today}
\def\lafecha{\@fecha}

\def\nombre#1{\gdef\@nombre{#1}}
\def\@nombre{No especific'o el nombre}
\def\elnombre{\@nombre}

\def\practicas#1{\gdef\@practica{#1}}
\def\@practica{No especifi\'o el n\'umero de pr\'actica}
\def\lapractica{\@practica}


% Esta macro convierte el numero de cuatrimestre a palabras
\newcommand{\cuatrimestreLindo}{
  \ifthenelse{\equal{\elcuatrimestre}{1}}
  {Primer cuatrimestre}
  {\ifthenelse{\equal{\elcuatrimestre}{2}}
  {Segundo cuatrimestre}
  {Verano}}
}


\newcommand{\depto}{{UBA -- Facultad de Ciencias Exactas y Naturales --
      Departamento de Computaci\'on}}

\newcommand{\titulopractica}{
  \centerline{\depto}
  \vspace{1ex}
  \centerline{{\Large\lamateria}}
  \vspace{0.5ex}
  \centerline{\cuatrimestreLindo de \elanio}
  \vspace{2ex}
  \centerline{{\huge Pr\'actica \lapractica -- \elnombre}}
  \vspace{5ex}
  \arreglarincisos
  \newcounter{ejercicio}
  \newenvironment{ejercicio}{\stepcounter{ejercicio}\textbf{Ejercicio
      \theejercicio}%
    \renewcommand\@currentlabel{\theejercicio}%
  }{\vspace{0.2cm}}
}  


\newcommand{\titulotp}{
  \centerline{\depto}
  \vspace{1ex}
  \centerline{{\Large\lamateria}}
  \vspace{0.5ex}
  \centerline{\cuatrimestreLindo de \elanio}
  \vspace{0.5ex}
  \centerline{\lafecha}
  \vspace{2ex}
  \centerline{{\huge\elnombre}}
  \vspace{5ex}
}


%practicas
\newcommand{\practica}[2]{%
    \title{Pr\'actica #1 \\ #2}
    \author{Algoritmos y Estructuras de Datos I}
    \date{Segundo Cuatrimestre 2015}

    \maketitlepractica{#1}{#2}
}

\newcommand \maketitlepractica[2] {%
\begin{center}
\begin{tabular}{r cr}
 \begin{tabular}{c}
{\large\bf\textsf{\ Algoritmos y Estructuras de Datos I\ }}\\ 
Segundo Cuatrimestre 2015\\
\title{\normalsize Gu\'ia Pr\'actica #1 \\ \textbf{#2}}\\
\@title
\end{tabular} &
\begin{tabular}{@{} p{1.6cm} @{}}
\includegraphics[width=1.6cm]{logodpt.jpg}
\end{tabular} &
\begin{tabular}{l @{}}
 \emph{Departamento de Computaci\'on} \\
 \emph{Facultad de Ciencias Exactas y Naturales} \\
 \emph{Universidad de Buenos Aires} \\
\end{tabular} 
\end{tabular}
\end{center}

\bigskip
}


% Simbolos varios

\newcommand{\ent}{\ensuremath{\mathbb{Z}}}
\newcommand{\float}{\ensuremath{\mathbb{R}}}
\newcommand{\bool}{\ensuremath{\mathsf{Bool}}}
\newcommand{\True}{\ensuremath{\mathrm{True}}}
\newcommand{\False}{\ensuremath{\mathrm{False}}}
\newcommand{\Then}{\ensuremath{\rightarrow}}
\newcommand{\Iff}{\ensuremath{\leftrightarrow}}
\newcommand{\implica}{\ensuremath{\longrightarrow}}
\newcommand{\IfThenElse}[3]{\ensuremath{\mathsf{if}\ #1\ \mathsf{then}\ #2\ \mathsf{else}\ #3}}


\newcommand{\rango}[2]{[#1\twodots#2]}
\newcommand{\comp}[2]{[\,#1\,|\,#2\,]}

\newcommand{\rangoac}[2]{(#1\twodots#2]}
\newcommand{\rangoca}[2]{[#1\twodots#2)}
\newcommand{\rangoaa}[2]{(#1\twodots#2)}

%ejercicios
\newtheorem{exercise}{Ejercicio}
\newenvironment{ejercicio}{\begin{exercise}\rm}{\end{exercise} \vspace{0.2cm}}
\newenvironment{items}{\begin{enumerate}[i)]}{\end{enumerate}}
\newenvironment{subitems}{\begin{enumerate}[a)]}{\end{enumerate}}
\newcommand{\sugerencia}[1]{\noindent \textbf{Sugerencia:} #1}

%tipos basicos
\newcommand{\rea}{\ensuremath{\mathsf{Float}}}
\newcommand{\cha}{\ensuremath{\mathsf{Char}}}

\newcommand{\mcd}{\mathrm{mcd}}
\newcommand{\prm}[1]{\ensuremath{\mathsf{prm}(#1)}}
\newcommand{\sgd}[1]{\ensuremath{\mathsf{sgd}(#1)}}

%listas
\newcommand{\TLista}[1]{[#1]}
\newcommand{\lvacia}{\ensuremath{[\ ]}}
\newcommand{\lv}{\ensuremath{[\ ]}}
\newcommand{\longitud}[1]{\left| #1 \right|}
\newcommand{\cons}[1]{\ensuremath{\mathsf{cons}}(#1)}
\newcommand{\indice}[1]{\ensuremath{\mathsf{indice}}(#1)}
\newcommand{\conc}[1]{\ensuremath{\mathsf{conc}}(#1)}
\newcommand{\cab}[1]{\ensuremath{\mathsf{cab}}(#1)}
\newcommand{\cola}[1]{\ensuremath{\mathsf{cola}}(#1)}
\newcommand{\sub}[1]{\ensuremath{\mathsf{sub}}(#1)}
\newcommand{\en}[1]{\ensuremath{\mathsf{en}}(#1)}
\newcommand{\cuenta}[2]{\mathsf{cuenta}\ensuremath{(#1, #2)}}
\newcommand{\suma}[1]{\mathsf{suma}(#1)}
\newcommand{\twodots}{\ensuremath{\mathrm{..}}}
\newcommand{\masmas}{\ensuremath{++}}

% Acumulador
\newcommand{\acum}[1]{\ensuremath{\mathsf{acum}}(#1)}
\newcommand{\acumselec}[3]{\ensuremath{\mathrm{acum}(#1 |  #2, #3)}}

% \selector{variable}{dominio}
\newcommand{\selector}[2]{#1~\ensuremath{\leftarrow}~#2}
\newcommand{\selec}{\ensuremath{\leftarrow}}


\newenvironment{problema}[4][res]{%
  % El parametro 1 (opcional) es el nombre del resultado
  % El parametro 2 es el nombre del problema
  % El parametro 3 son los parametros
  % El parametro 4 es el tipo del resultado
  % Preambulo del ambiente problema
  % Tenemos que definir los comandos requiere, asegura, modifica y aux
  \newcommand{\requiere}[2][]{%
    {\normalfont\bfseries\ttfamily requiere}%
    \ifthenelse{\equal{##1}{}}{}{\ {\normalfont\ttfamily ##1} :}\ %
    \ensuremath{##2}%
    {\normalfont\bfseries\,;\par}%
  }
  \newcommand{\asegura}[2][]{%
    {\normalfont\bfseries\ttfamily asegura}%
    \ifthenelse{\equal{##1}{}}{}{\ {\normalfont\ttfamily ##1} :}\
    \ensuremath{##2}%
    {\normalfont\bfseries\,;\par}%
  }
  \newcommand{\modifica}[1]{%
    {\normalfont\bfseries\ttfamily modifica\ }%
    \ensuremath{##1}%
    {\normalfont\bfseries\,;\par}%
  }
  \renewcommand{\aux}[4]{%
    {\normalfont\bfseries\ttfamily aux\ }%
    {\normalfont\ttfamily ##1}%
    \ifthenelse{\equal{##2}{}}{}{\ (##2)}\ : ##3\, = \ensuremath{##4}%
    {\normalfont\bfseries\,;\par}%
  }
  \newcommand{\res}{#1}
  \vspace{1ex}
  \noindent
  \encabezadoDeProblema{#1}{#2}{#3}{#4}
  % Abrimos la llave
  \{\par%
  \tocarEspacios
}
% Ahora viene el cierre del ambiente problema
{
  % Cerramos la llave
  \noindent\}
  \vspace{1ex}
}


  \newcommand{\aux}[4]{%
    {\normalfont\bfseries\ttfamily aux\ }%
    {\normalfont\ttfamily #1}%
    \ifthenelse{\equal{#2}{}}{}{\ (#2)}\ : #3\, = \ensuremath{#4}%
    {\normalfont\bfseries\,;\par}%
  }


\newcommand{\pre}[1]{\textsf{pre}\ensuremath{(#1)}}

\newcommand{\problemanom}[1]{\textsf{#1}}
\newcommand{\problemail}[3]{\textsf{problema #1}\ensuremath{(#2) = #3}}
\newcommand{\problemailsinres}[2]{\textsf{problema #1}\ensuremath{(#2)}}
\newcommand{\requiereil}[2]{\textsf{requiere #1: }\ensuremath{#2}}
\newcommand{\asegurail}[2]{\textsf{asegura #1: }\ensuremath{#2}}
\newcommand{\modificail}[1]{\textsf{modifica }\ensuremath{#1}}
\newcommand{\auxil}[2]{\textsf{aux }\ensuremath{#1 = #2}}
\newcommand{\auxilc}[4]{\textsf{aux }\ensuremath{#1( #2 ): #3 = #4}}
\newcommand{\auxnom}[1]{\textsf{aux }\ensuremath{#1}}

\newcommand{\comentario}[1]{{/*\ #1\ */}}

\newcommand{\nom}[1]{\ensuremath{\mathsf{#1}}}

% -----------------
% Tipos compuestos
% -----------------

\newcommand{\Pred}[1]{\mathit{#1}}
\newcommand{\TSet}[1]{\textsf{Conjunto}\ensuremath{\langle #1 \rangle}}
\newcommand{\TSetFinito}[1]{\textsf{Conjunto}\ensuremath{\langle #1 \rangle}}
\newcommand{\TRac}{\tiponom{Racional}}
\newcommand{\TVec}{\tiponom{Vector}}
\newcommand{\Func}[1]{\mathrm{#1}}
\newcommand{\cardinal}[1]{\left| #1 \right|}


\newcommand{\sinonimo}[2]{%
  \noindent%
  {\normalfont\bfseries\ttfamily tipo\ }%
  #1\ =\ #2%
  {\normalfont\bfseries\,;\par}
}

\newcommand{\enum}[2]{%
  \noindent%
  {\normalfont\bfseries\ttfamily tipo\ }%
  #1\ =\ #2%
  {\normalfont\bfseries\,;\par}
}

%~ \newenvironment{tipo}[1]{%
    %~ \vspace{0.2cm}
    %~ \textsf{tipo #1}\ensuremath{\{}\\
    %~ \begin{tabular}[l]{p{0.02\textwidth} p{0.02\textwidth} p{0.82 \textwidth}}
%~ }{%
    %~ \end{tabular}
%~ 
    %~ \ensuremath{\}}
    %~ \vspace{0.15cm}
%~ }
%~ 

\newenvironment{tipo}[2][]{%
  % Preambulo del ambiente tipo
  % Tenemos que definir los comandos observador (con requiere) y aux
  \newcommand{\observador}[3]{%
    {\normalfont\bfseries\ttfamily observador\ }%
    {\normalfont\ttfamily ##1}%
    \ifthenelse{\equal{##2}{}}{}{\ (##2)}\ : ##3%
    {\normalfont\bfseries\,;\par}%
  }
  \newcommand{\requiere}[2][]{{%
    \addtolength{\leftskip}{3em}%
    \setlength{\parindent}{-2em}%
    {\normalfont\bfseries\ttfamily requiere}%
    \ifthenelse{\equal{##1}{}}{}{\ {\normalfont\ttfamily ##1} :}\ 
    \ensuremath{##2}%
    {\normalfont\bfseries\,;\par}}
  }
  \newcommand{\explicacion}[1]{{%
    \addtolength{\leftskip}{3em}%
    \setlength{\parindent}{-2em}%
    \par \hspace{2.3em} ##1 %
    {\par}
    }
  }
  \newcommand{\invariante}[2][]{%
    {\normalfont\bfseries\ttfamily invariante}%
    \ifthenelse{\equal{##1}{}}{}{\ {\normalfont\ttfamily ##1} :}\ 
    \ensuremath{##2}%
    {\normalfont\bfseries\,;\par}%
  }
  \renewcommand{\aux}[4]{%
    {\normalfont\bfseries\ttfamily aux\ }%
    {\normalfont\ttfamily ##1}%
    \ifthenelse{\equal{##2}{}}{}{\ (##2)}\ : ##3\, = \ensuremath{##4}%
    {\normalfont\bfseries\,;\par}%
  }
  \vspace{1ex}
  \noindent
  \encabezadoDeTipo{#1}{#2}
  % Abrimos la llave
  \{\par%
  \tocarEspacios
}
% Ahora viene el cierre del ambiente tipo
{
  % Cerramos la llave
  \noindent\}
  \vspace{1ex}
}


%~ \newcommand{\observador}[3]{%
    %~ & \multicolumn{2}{p{0.85\textwidth}}{\textsf{observador #1}\ensuremath{(#2):#3}}\\%
    %~ }
    
%~ \newcommand{\observador}[3]{%
    %~ {\normalfont\bfseries\ttfamily observador\ }%
    %~ {\normalfont\ttfamily ##1}%
    %~ \ifthenelse{\equal{##2}{}}{}{\ (##2)}\ : ##3%
    %~ {\normalfont\bfseries\,;\par}%
%~ }
    

%~ \newcommand{\observadorconreq}[3]{
    %~ & \multicolumn{2}{p{0.85\textwidth}}{\textsf{observador #1}\ensuremath{(#2):#3 \{}}\\
%~ }
%~ \newcommand{\observadorconreqfin}{
    %~ & \multicolumn{2}{p{0.85\textwidth}}{\ensuremath{\}}}\\
%~ }
%~ \newcommand{\obsrequiere}[2][]{& & \textsf{requiere #1: }\ensuremath{#2};\\}
%~ 
%~ \newcommand{\explicacion}[1]{&& #1 \\}
%~ \newcommand{\invariante}[2][]{%
    %~ & \multicolumn{2}{p{0.85\textwidth}}{\textsf{invariante #1: }\ensuremath{#2}}\\%
%~ }
%~ \newcommand{\auxinvariante}[2]{
    %~ & \multicolumn{2}{p{0.85\textwidth}}{\textsf{aux }\ensuremath{#1 = #2}};\\
%~ }
%~ \newcommand{\auxiliar}[4]{
    %~ & \multicolumn{2}{p{0.85\textwidth}}{\textsf{aux }\ensuremath{#1(#2): #3 = #4}};\\
%~ }

\newcommand{\tiponom}[1]{\ensuremath{\mathsf{#1}}\xspace}
\newcommand{\obsnom}[1]{\ensuremath{\mathsf{#1}}}

% -----------------
% Ecuaciones de terminacion en funcional
% -----------------

\newenvironment{ecuaciones}{%
    $$
    \begin{array}{l @{\ /\ (} l @{,\ } l @{)\ =\ } l}
}{%
    \end{array}
    $$
}




\newcommand{\ecuacion}[4]{#1 & #2 & #3 & #4\\}

\newcommand{\concat}{\nom{concat}}

% Listas por comprension. El primer parametro es la expresion y el
% segundo tiene los selectores y las condiciones.
%*\newcommand{\comp}[2]{[\,#1\,|\,#2\,]}























% En las practicas/parciales usamos numeros arabigos para los ejercicios.
% Aca cambiamos los enumerate comunes para que usen letras y numeros
% romanos
\newcommand{\arreglarincisos}{%
  \renewcommand{\theenumi}{\alph{enumi}}
  \renewcommand{\theenumii}{\roman{enumii}}
  \renewcommand{\labelenumi}{\theenumi)}
  \renewcommand{\labelenumii}{\theenumii)}
}





%%%%%%%%%%%%%%%%%%%%%%%%%%%%%% PARCIAL %%%%%%%%%%%%%%%%%%%%%%%%
\let\@xa\expandafter
\newcommand{\tituloparcial}{\centerline{\depto -- \lamateria}
  \centerline{\elnombre -- \lafecha}%
  \setlength{\TPHorizModule}{10mm} % Fija las unidades de textpos
  \setlength{\TPVertModule}{\TPHorizModule} % Fija las unidades de
                                % textpos
  \arreglarincisos
  \newcounter{total}% Este contador va a guardar cuantos incisos hay
                    % en el parcial. Si un ejercicio no tiene incisos,
                    % cuenta como un inciso.
  \newcounter{contgrilla} % Para hacer ciclos
  \newcounter{columnainicial} % Se van a usar para los cline cuando un
  \newcounter{columnafinal}   % ejercicio tenga incisos.
  \newcommand{\primerafila}{}
  \newcommand{\segundafila}{}
  \newcommand{\rayitas}{} % Esto va a guardar los \cline de los
                          % ejercicios con incisos, asi queda mas bonito
  \newcommand{\anchodegrilla}{20} % Es para textpos
  \newcommand{\izquierda}{7} % Estos dos le dicen a textpos donde colocar
  \newcommand{\abajo}{2}     % la grilla
  \newcommand{\anchodecasilla}{0.4cm}
  \setcounter{columnainicial}{1}
  \setcounter{total}{0}
  \newcounter{ejercicio}
  \setcounter{ejercicio}{0}
  \renewenvironment{ejercicio}[1]
  {%
    \stepcounter{ejercicio}\textbf{\noindent Ejercicio \theejercicio. [##1
      puntos]}% Formato
    \renewcommand\@currentlabel{\theejercicio}% Esto es para las
                                % referencias
    \newcommand{\invariante}[2]{%
      {\normalfont\bfseries\ttfamily invariante}%
      \ ####1\hspace{1em}####2%
    }%
    \renewcommand{\problema}[5][result]{
      \encabezadoDeProblema{####1}{####2}{####3}{####4}\hspace{1em}####5}%
  }% Aca se termina el principio del ejercicio
  {% Ahora viene el final
    % Esto suma la cantidad de incisos o 1 si no hubo ninguno
    \ifthenelse{\equal{\value{enumi}}{0}}
    {\addtocounter{total}{1}}
    {\addtocounter{total}{\value{enumi}}}
    \ifthenelse{\equal{\value{ejercicio}}{1}}{}
    {
      \g@addto@macro\primerafila{&} % Si no estoy en el primer ej.
      \g@addto@macro\segundafila{&}
    }
    \ifthenelse{\equal{\value{enumi}}{0}}
    {% No tiene incisos
      \g@addto@macro\primerafila{\multicolumn{1}{|c|}}
      \bgroup% avoid overwriting somebody else's value of \tmp@a
      \protected@edef\tmp@a{\theejercicio}% expand as far as we can
      \@xa\g@addto@macro\@xa\primerafila\@xa{\tmp@a}%
      \egroup% restore old value of \tmp@a, effect of \g@addto.. is
      
      \stepcounter{columnainicial}
    }
    {% Tiene incisos
      % Primero ponemos el encabezado
      \g@addto@macro\primerafila{\multicolumn}% Ahora el numero de items
      \bgroup% avoid overwriting somebody else's value of \tmp@a
      \protected@edef\tmp@a{\arabic{enumi}}% expand as far as we can
      \@xa\g@addto@macro\@xa\primerafila\@xa{\tmp@a}%
      \egroup% restore old value of \tmp@a, effect of \g@addto.. is
      % global 
      % Ahora el formato
      \g@addto@macro\primerafila{{|c|}}%
      % Ahora el numero de ejercicio
      \bgroup% avoid overwriting somebody else's value of \tmp@a
      \protected@edef\tmp@a{\theejercicio}% expand as far as we can
      \@xa\g@addto@macro\@xa\primerafila\@xa{\tmp@a}%
      \egroup% restore old value of \tmp@a, effect of \g@addto.. is
      % global 
      % Ahora armamos la segunda fila
      \g@addto@macro\segundafila{\multicolumn{1}{|c|}{a}}%
      \setcounter{contgrilla}{1}
      \whiledo{\value{contgrilla}<\value{enumi}}
      {%
        \stepcounter{contgrilla}
        \g@addto@macro\segundafila{&\multicolumn{1}{|c|}}
        \bgroup% avoid overwriting somebody else's value of \tmp@a
        \protected@edef\tmp@a{\alph{contgrilla}}% expand as far as we can
        \@xa\g@addto@macro\@xa\segundafila\@xa{\tmp@a}%
        \egroup% restore old value of \tmp@a, effect of \g@addto.. is
        % global 
      }
      % Ahora armo las rayitas
      \setcounter{columnafinal}{\value{columnainicial}}
      \addtocounter{columnafinal}{-1}
      \addtocounter{columnafinal}{\value{enumi}}
      \bgroup% avoid overwriting somebody else's value of \tmp@a
      \protected@edef\tmp@a{\noexpand\cline{%
          \thecolumnainicial-\thecolumnafinal}}%
      \@xa\g@addto@macro\@xa\rayitas\@xa{\tmp@a}%
      \egroup% restore old value of \tmp@a, effect of \g@addto.. is
      \setcounter{columnainicial}{\value{columnafinal}}
      \stepcounter{columnainicial}
    }
    \setcounter{enumi}{0}%
    \vspace{0.2cm}%
  }%
  \newcommand{\tercerafila}{}
  \newcommand{\armartercerafila}{
    \setcounter{contgrilla}{1}
    \whiledo{\value{contgrilla}<\value{total}}
    {\stepcounter{contgrilla}\g@addto@macro\tercerafila{&}}
  }
  \newcommand{\grilla}{%
    \g@addto@macro\primerafila{&\textbf{TOTAL}}
    \g@addto@macro\segundafila{&}
    \g@addto@macro\tercerafila{&}
    \armartercerafila
    \ifthenelse{\equal{\value{total}}{\value{ejercicio}}}
    {% No hubo incisos
      \begin{textblock}{\anchodegrilla}(\izquierda,\abajo)
        \begin{tabular}{|*{\value{total}}{p{\anchodecasilla}|}c|}
          \hline
          \primerafila\\
          \hline
          \tercerafila\\
          \tercerafila\\
          \hline
        \end{tabular}
      \end{textblock}
    }
    {% Hubo incisos
      \begin{textblock}{\anchodegrilla}(\izquierda,\abajo)
        \begin{tabular}{|*{\value{total}}{p{\anchodecasilla}|}c|}
          \hline
          \primerafila\\
          \rayitas
          \segundafila\\
          \hline
          \tercerafila\\
          \tercerafila\\
          \hline
        \end{tabular}
      \end{textblock}
    }
  }%
  \vspace{0.4cm}
  \textbf{Nro. de orden:}
  
  \textbf{LU:}
  
  \textbf{Apellidos:}
  
  \textbf{Nombres:}
  \vspace{0.5cm}
}



% AMBIENTE CONSIGNAS
% Se usa en el TP para ir agregando las cosas que tienen que resolver
% los alumnos.
% Dentro del ambiente hay que usar \item para cada consigna

\newcounter{consigna}
\setcounter{consigna}{0}

\newenvironment{consignas}{%
  \newcommand{\consigna}{\stepcounter{consigna}\textbf{\theconsigna.}}%
  \renewcommand{\ejercicio}[1]{\item ##1 }
  \renewcommand{\problema}[5][result]{\item
    \encabezadoDeProblema{##1}{##2}{##3}{##4}\hspace{1em}##5}%
  \newcommand{\invariante}[2]{\item%
    {\normalfont\bfseries\ttfamily invariante}%
    \ ##1\hspace{1em}##2%
  }
  \renewcommand{\aux}[4]{\item%
    {\normalfont\bfseries\ttfamily aux\ }%
    {\normalfont\ttfamily ##1}%
    \ifthenelse{\equal{##2}{}}{}{\ (##2)}\ : ##3 \hspace{1em}##4%
  }
  % Comienza la lista de consignas
  \begin{list}{\consigna}{%
      \setlength{\itemsep}{0.5em}%
      \setlength{\parsep}{0cm}%
    }
}%
{\end{list}}



% para decidir si usar && o ^
\newcommand{\y}[0]{\ensuremath{\land}}

% macros de correctitud
\newcommand{\semanticComment}[2]{#1 \ensuremath{#2};}
\newcommand{\namedSemanticComment}[3]{#1 #2: \ensuremath{#3};}


\newcommand{\local}[1]{\semanticComment{local}{#1}}

\newcommand{\vale}[1]{\semanticComment{vale}{#1}}
\newcommand{\valeN}[2]{\namedSemanticComment{vale}{#1}{#2}}
\newcommand{\impl}[1]{\semanticComment{implica}{#1}}
\newcommand{\implN}[2]{\namedSemanticComment{implica}{#1}{#2}}
\newcommand{\estado}[1]{\semanticComment{estado}{#1}}

\newcommand{\invarianteCN}[2]{\namedSemanticComment{invariante}{#1}{#2}}
\newcommand{\invarianteC}[1]{\semanticComment{invariante}{#1}}
\newcommand{\varianteCN}[2]{\namedSemanticComment{variante}{#1}{#2}}
\newcommand{\varianteC}[1]{\semanticComment{variante}{#1}}
% Macros especificas para especificar problemas en AyEDI
\usepackage{caratula} % Se puede descargar en ~> https://github.com/bcardiff/dc-tex

% Aca solo vamos a poner el esqueleto del documento, pero no vamos a especificar nada.

\begin{document} % Todo lo que escribamos a partir de aca va a aparecer en el documento.

% Completar los datos de la caratula
\titulo{TPE - Agricultura con drones} 
\fecha{\today}
\materia{Algoritmos y Estructuras de Datos I}
\grupo{Grupo ?}

% Completar con cuantos integrantes quieran :)
\integrante{Apellido, Nombre1}{592/15}{email1@dominio.com}
\integrante{Apellido, Nombre2}{002/01}{email2@dominio.com}
\integrante{Apellido, Nombre3}{003/01}{email3@dominio.com}
\integrante{Apellido, Nombre4}{004/01}{email4@dominio.com}

\maketitle

\section{Tipos}

\newcommand{\id}{Id}
\newcommand{\carga}{Carga}
\newcommand{\ancho}{Ancho}
\newcommand{\largo}{Largo}
\sinonimo{\id}{\ent}
\sinonimo{\carga}{\ent}
\sinonimo{\ancho}{\ent}
\sinonimo{\largo}{\ent}
\enum{Parcela}{Cultivo, Granero, Casa}
\enum{Producto}{Fertilizante, Plaguicida, PlaguicidaBajoConsumo, Herbicida, HerbicidaLargoAlcance}
\enum{EstadoCultivo}{Reci\'enSembrado, EnCrecimiento, ListoParaCosechar, ConMaleza, ConPlaga, NoSensado}


\section{Campo}

\begin{tipo}{Campo}
	\observador{dimensiones}{c: Campo}{(Ancho, Largo)}
	\observador{contenido}{c: Campo, i, j: \ent}{Parcela}
		\requiere[enRango]{0 \leq i < prm(dimensiones(c)) \land 0 \leq j < sgd(dimensiones(c))}
	\medskip
	\invariante[dimensionesValidas]{prm(dimensiones(c)) > 0 \land sgd(dimensiones(c)) > 0}
	\invariante[unaSolaCasa]{|[(i, j) | i \selec \rangoca{0}{prm(dimensiones(c))},  j \selec \rangoca{0}{sgd(dimensiones(c))}, \\ contenido(c, i, j) == Casa]| == 1}
	\invariante[unSoloGranero]{|[(i, j) | i \selec \rangoca{0}{prm(dimensiones(c))},  j \selec \rangoca{0}{sgd(dimensiones(c))}, \\ contenido(c, i, j) == Granero]| == 1}
	\invariante[algoDeCultivo]{|[(i, j) | i \selec \rangoca{0}{prm(dimensiones(c))},  j \selec \rangoca{0}{sgd(dimensiones(c))}, \\ contenido(c, i, j) == Cultivo]| \geq 1}
	\invariante[posicionesAlcanzables]{posicionesAlcanzablesEn100(c)}

\end{tipo}

\noindent \aux{posicionesAlcanzablesEn100}{c: Campo}{\bool}{\\alcanzableEn100(posicionGranero(c), prm(dimensiones(c)), sgd(dimensiones(c)))}


\begin{problema}{crearC}{posG, posC: (\ent, \ent)}{Campo}
\requiere{posG \neq posC}
\requiere{posNoNegativas(posG) \land posNoNegativas(posC)}
\requiere{distancia(posG, (0,0)) \leq 100}
\requiere{distancia(posG, posC) \leq 100}

\asegura {contenido(c, prm(posG), sgd(posG)) == Granero}
\asegura {contenido(c, prm(posC), sgd(posC)) == Casa}

\end{problema}

\begin{problema}{dimensionesC}{c: Campo}{(Ancho, Largo))}
\end{problema}

\begin{problema}{contenidoC}{c: Campo, i, j: \ent}{Parcela}
\end{problema}


\newpage

\section{Drone}

\begin{tipo}{Drone}
	\observador{id}{d: Drone}{\id}
	\observador{bateria}{d: Drone}{\carga}
	\observador{enVuelo}{d: Drone}{\bool}
	\observador{vueloRealizado}{d: Drone}{[(\ent, \ent)]}
	\observador{posicionActual}{d: Drone}{(\ent, \ent)}
	\observador{productosDisponibles}{d: Drone}{[Producto]}
	\medskip
	\invariante[vuelosOk]{\\ enVuelo(d) \Rightarrow (\longitud{vueloRealizado(d)} > 0 \land posicionActual(d) == vueloRealizado(d)_{\longitud{vueloRealizado(d)}-1} \land \\ posicionesPositivas(d) \land movimientosOK(d)) \land	\neg enVuelo(d) \Rightarrow \longitud{vueloRealizado(d)} == 0 }
	\invariante[bateriaOk]{0 \leq bateria(d) \leq 100}
\end{tipo}

\noindent \aux{posicionesPositivas}{d: Drone}{\bool}{(\forall i \selec \rangoca{0}{\longitud{vueloRealizado(d)}})  prm(vueloRealizado(d)_i) \geq 0 \land \\ sgd(vueloRealizado(d)_i \geq 0}

\noindent \aux{movimientosOK}{d: Drone}{\bool}{(\forall i \selec \rangoca{1}{\longitud{vueloRealizado(d)}}) \\ prm(vueloRealizado(d)_i) == prm(vueloRealizado(d)_{i-1}) \land (sgd(vueloRealizado(d)_i) == sgd(vueloRealizado(d)_{i-1}) - 1 \lor sgd(vueloRealizado(d)_i) == sgd(vueloRealizado(d)_{i-1}) + 1) \lor sgd(vueloRealizado(d)_i) == sgd(vueloRealizado(d)_{i-1}) \\ \land (prm(vueloRealizado(d)_i) == prm(vueloRealizado(d)_{i-1}) - 1 \lor prm(vueloRealizado(d)_{i-1}) == prm(vueloRealizado(d)_{i-1}) + 1)}


% Problema 4
\begin{problema}{crearD}{id: \ent, ps: [Producto]}{Drone}
\asegura{id(\res) == id }
\asegura[bateriaAlMaximo]{bateria(\res) == 100}
\asegura{mismos(\res,\hspace{1mm} ps)}
\asegura{enVuelo(\res) == false}
\end{problema}

% Problema 5
\begin{problema}{idD}{d: Drone}{\ent}
\asegura{\res == id(d)}
\end{problema}

% Problema 6
\begin{problema}{bateriaD}{d: Drone}{\ent}
\asegura{\res == bateria(d)}
\end{problema}

% Problema 7
\begin{problema}{enVueloD}{d: Drone}{\bool}
\asegura{\res == enVuelo(d)}
\end{problema}

% Problema 8
\begin{problema}{vueloRealizadoD}{d: Drone}{[(\ent, \ent)]}
\asegura{\res == vueloRealizado(d)}
\end{problema}

% Problema 9
\begin{problema}{posicionActualD}{d: Drone}{(\ent, \ent)}
\asegura{\res == posicionActual(d)}
\end{problema}

% Problema 10
\begin{problema}{productosDisponiblesD}{d: Drone}{[Producto]}
\asegura{mismos(\res, productosDisponibles(d))}
\end{problema}

% Problema 11
\begin{problema}{vueloEscaleradoD}{d: Drone}{\bool}

\asegura{\res == (enVuelo(d)\newline \land comoMuchoDosVeces(posicionesEnX(vueloRealizado(d)))
\newline \land estaOrdenada(posicionesEnX(vueloRealizado(d)))\newline \land comoMuchoDosVeces(posicionesEnY(vueloRealizado(d)))\newline \land estaOrdenada(posicionesEnY(vueloRealizado(d) ) ) \hspace{1mm})}
\end{problema}

\newpage

% Problema 12
\begin{problema}{vuelosCruzadosD}{ds: [Drone]}{[((\ent, \ent), \ent)]}
\requiere[todosValidos]{(\forall \selector{d}{ds}) enVuelo(d)}
\requiere{(\forall \selector{i}{\rangoca{0}{|ds|}}) \newline 
|vueloRealizado(ds[0])| ==  | vueloRealizado(ds[i]) |}
\asegura{mismos(\res, crucesTotales(posicionesPorMomentos(ds)))}
\asegura{estaOrdenada(listaSegundosElementos(\res))}
\end{problema}


\newpage

\section{Sistema}

\begin{tipo}{Sistema}
	\observador{campo}{s: Sistema}{Campo}
	\observador{estadoDelCultivo}{s: Sistema, i, j: \ent}{EstadoCultivo}
		\requiere{enRango(dimensiones(s), i, j) \land contenido(campo(s), i,j) == Cultivo}
	\observador{enjambreDrones}{s: Sistema}{[Drone]}

	\medskip

	\invariante[identificadoresUnicos]{sinRepetidos(\comp{id(d)}{d \selec enjambreDrones(s)})}
	\invariante[unoPorParcela]{(\forall d, d' \selec dronesEnVuelo(s), id(d) \neq id(d')) posicionActual(d) \neq posicionActual(d')}
	\invariante[siNoVuelanEstanEnGranero]{(\forall d \selec enjambreDrones(s), \neg enVuelo(d)) \\ posicionActual(d) == posicionGranero(campo(s))}
	\invariante[siEstanEnVueloElVueloEstaEnRango]{(\forall d \selec dronesEnVuelo(s)) (\forall v \selec vueloRealizado(d))\\ enRango(dimensiones(campo(s), prm(v), sgd(v))} 
\end{tipo}

\aux{dronesEnVuelo}{s: Sistema}{[Drone]}{\comp{d}{d \selec enjambreDrones(s), enVuelo(d)}}


% Problema 13
\begin{problema}{crearS}{c: Campo, ds: [Drone]}{Sistema}
\requiere{sinRepetidos(listaIds(ds))}
\asegura {c == campo(res)}
\asegura[todoCultivoEstaNoCensado]{(\forall \selector{i}{\rangoca{0}{dameAncho(c)}} \hspace{1mm} \selector{j}{\rangoca{0}{dameLargo(c)}})\hspace{1mm}((contenido(c, i, j) == Cultivo) \implica (estadoDelCultivo(\res, i, j) == NoCensado))  )}
\asegura{|ds| == |enjambreDrones(res)|}
\asegura{(\forall \selector{d}{ds}) \hspace{1mm} id(d) \in listaIds(enjambreDrones(res)) )}
\asegura{(\forall \selector{x}{enjambreDrones(res)}) \hspace{1mm}bateria(x) == 100}
\asegura{(\forall \selector{x}{enjambreDrones(res)}) \hspace{1mm}enVuelo(x) == False}
\asegura[productosNoCambiaron]{(\forall \selector{x}{enjambreDrones(res)}, \selector{d}{ds}, id(x) == id(d))\newline mismos(productosDisponibles(x), productosDisponibles(d))}
\asegura[todosEnGranero]{(\forall \selector{x}{enjambreDrones(res)}) \newline
\hspace{1mm}contenido(c, prm(posActual(x)), sgd(posActual(x)) == Granero}

\end{problema}

% Problema 14
\begin{problema}{campoS}{s: Sistema}{Campo}
\asegura{\res == campo(s)}
\end{problema}

% Problema 15
% MODIFICACION: Necesitabamos perdir que la posicion sea cultivo. Si no lo era, estado de cultivo explotaba.
\begin{problema}{estadoDelCultivoS}{s: Sistema, i, j: \ent}{EstadoCultivo}
\requiere{enRango(dimensiones(campo(s)), i, j)}
\requiere{contenido(campo(s), i ,j) == Cultivo}
\asegura{\res == estadoDelCultivo(s, i, j)}
\end{problema}

% Problema 16
\begin{problema}{enjambreDronesS}{s: Sistema}{[Drone]}
\asegura{mismos(\res, enjambreDrones(s))}
\end{problema}

% Problema 17
\begin{problema}{crecerS}{s: Sistema}{}
\modifica{s}
\asegura{campo(pre(s)) == campo(s)}
\asegura{mismos(enjambreDrones(pre(s)), enjambreDrones(s))}

\asegura{(\forall\selector{i}{\rangoca{0}{dameAncho(campo(s))}},\selector{j}{\rangoca{0}{dameLargo(campo(s))}}, contenido(campo(pre(s)), i, j) == Cultivo)\newline
((estadoDelCultivo(pre(s),i,j) == RecienSembrado) \implica (estadoDelCultivo(s, i, j) == EnCrecimiento))\newline
\land ((estadoDelCultivo(pre(s),i,j) == EnCrecimiento) \implica (estadoDelCultivo(s, i, j) == ListoParaCosechar))\newline
\land ((estadoDelCultivo(pre(s), i, j) \neq RecienSembrado\hspace{1mm} \land \hspace{1mm} estadoDelCultivo(pre(s), i, j)  \neq EnCrecimiento) \newline \implica estadoDelCultivo(pre(s), i, j) == estadoDelCultivo(s, i, j))}

\end{problema}

\newpage

% Problema 18
\begin{problema}{seVinoLaMalezaS}{s: Sistema, ps: [(\ent, \ent)]}{}
\requiere {(\forall \selector{p}{ps})\hspace{1mm}enRango(dimensiones(campo(s)), prm(p), sgd(p))}
\requiere {(\forall \selector{p}{ps})\hspace{1mm}contenido(campo(s), prm(p), sgd(p) == Cultivo}
\modifica{s} 
\asegura{campo(pre(s)) == campo(s)}
\asegura{mismos(enjambreDrones(pre(s)), enjambreDrones(s))}
\asegura{(\forall \selector{p}{ps})\hspace{1mm} estadoDelCultivo(s, prm(p), sgd(p)) == ConMaleza}
\asegura{(\forall\selector{i}{\rangoca{0}{dameAncho(campo(s))}},\selector{j}{\rangoca{0}{dameLargo(campo(s))}}, contenido(campo(pre(s)), i, j) == Cultivo, (i,j) \not\in ps) \newline
estadoDelCultivo(pre(s), i ,j) == estadoDelCultivo(s, i, j)}

\end{problema}

% Problema 19
\begin{problema}{seExpandePlagaS}{s: Sistema}{}
\modifica{s}
\asegura{campo(pre(s)) == campo(s)}
\asegura{mismos(enjambreDrones(pre(s)), enjambreDrones(s))}

\asegura{(\forall\selector{i}{\rangoca{0}{dameAncho(campo(s))}}, \selector{j}{\rangoca{0}{dameLargo(campo(s))}}, contenido(campo(pre(s)), i, j == Cultivo)\newline
(((i,j)\in recibenPlaga(pre(s)) ) \implica \hspace{1mm}estadoDelCultivo(pre(s), i, j) == ConPlaga)\newline
\land \hspace{1mm} (((i, j) \not\in recibenPlaga(pre(s)))\implica \hspace{1mm}estadoDelCultivo(pre(s), i, j) == estadoDelCultivo(s,i,j)) }

\end{problema}

% Problema 20
\begin{problema}{despegarS}{s: Sistema, d: Drone}{}
\requiere{\neg enVuelo(d)}
\requiere{bateria(d) == 100}
\requiere{d \in enjambreDrones(s)}
\requiere[droneEnGranero]{contenido(campo(s), prm(\hspace{1mm}posicionActual(d)), \hspace{1mm}sgd(posicionActual(d))) == Granero}
\requiere{|adyacentesValidasGranero(s)| \geq 1}

\modifica{s}

\asegura[mismoCampo]{campo(pre(s)) == campo(s)}
\asegura[igualCantidaDeDrones]{|enjabreDrones(pre(s))| == |enjambreDrones(s)|}
\asegura[restoDronesIgual]{(\forall \selector{x}{enjambreDrones(pre(s))},\hspace{1mm}x \neq d)\hspace{3mm} x \in enajmbreDrones(s)}
\asegura[unDroneCambiado]{( \exists \selector{x}{enjambreDrones(s)})\newline \hspace{16cm} id(x) == id(d)\newline \hspace{16cm}\land bateria(x) == bateria(d) - 1 \newline \hspace{16cm} \land estaEnVuelo(x) \newline \hspace{16cm} \land posicionActual(x) \in adyacentesValidosGranero(s)\newline \hspace{16cm} \land vueloRealizado(x) == [posicionActual(x)] \newline \hspace{16cm} \land mismos(productosDisponibles(x), productosDisponibles(d))}
\asegura{noCambiaEstadoDeCultivos(s, pre(s))}
\end{problema}

% Problema 21
\begin{problema}{listoParaCosecharS}{s: Sistema}{\bool}
\asegura{\res == (\hspace{1mm} porcentajeCultivosListos(s) \geq 0.90 \hspace{1mm})}
\end{problema}

% Problema 22
\begin{problema}{aterrizarYCargarBateriaS}{s: Sistema, b: \ent}{}
\end{problema}

% Problema 23
\begin{problema}{fertilizarPorFilas}{s: Sistema}{}
\end{problema}

% Problema 24
% COMPLETAR!
\begin{problema}{volarYSensarS}{s: Sistema, d: Drone}{}
\requiere {d \in enjambreDrones (s)}
\requiere {bateria (d) \geq 1}
\modifica{s}
\asegura {campo (s) == campo (pre(s))}
\asegura {|enjambreDrones (pre(s))| == |enjambreDrones(s)|}
\asegura {(\forall \selector{x}{enjambreDrones (pre(s))} x \neq d )\hspace{1mm} x \in enajmbreDrones (s)}
\end{problema}


%\input{espec/ejersRecu}

\newpage

\section{Funciones Auxiliares}


\subsection{Campo}
% los aux del tipo campo
\aux{posNoNegativas}{pos: (\ent, \ent)}{\bool}{(prm(pos) \geq 0 \land sgd(pos) \geq 0) }

\aux{dameAncho}{c: Campo}{\ent}{prm(dimensiones(c))}
\aux{dameLargo}{c: Campo}{\ent}{sgd(dimensiones(c))}

\subsection{Drone}
% los aux del tipo drone

% NOTA(Jonathan): HAY PROBLEMAS QUE SE USAN EN 12 QUE ESTAN EN LA SECCION 11. ESTO ESTA BIEN PUES SE USAN EN AMBOS.
% HABRIA QUE DEJAR EN ESTA SECCION LAS AUXILIARES QUE USAMOS EN VARIOS
% LOS ESPECIFICOS DE PROBLEMAS DEJEMOSLOS EN SUS RESPECTIVOS PROBLEMAS.

% Problema 11

\aux{listaPrimerosElementos}{xs : [(T, S)]}{[T]}{[prm(v) \hspace{3mm} | \hspace{3mm} \selector{v}{xs}]}

\aux{listaSegundosElementos}{xs : [(T, S)]}{[S]}{[sgd(v) \hspace{3mm} | \hspace{3mm} \selector{v}{xs}]}

\aux{posicionesEnX}{xs : [(T, S)]}{[T]}{listaPrimerosElementos(xs)}

\aux{posicionesEnY}{xs : [(T, S)]}{[S]}{listaSegundosElementos(xs)}

\aux{estaOrdenada}{xs : [\ent]}{Bool}{siempreCreciente(xs) \lor siempreDecreciente(xs)}

\aux{siempreCreciente}{xs : [\ent]}{Bool}{(\forall \selector{i}{\rangoca{0}{|xs|-1}})\hspace{1mm} xs[i] \leq xs[i+1]}

\aux{siempreDecreciente}{xs : [\ent]}{Bool}{(\forall \selector{i}{[0.. |xs|-1)})\hspace{1mm} xs[i] \geq xs[i+1]}

\aux{comoMuchoDosVeces}{xs : [\ent]}{Bool}{(\forall \selector{x}{xs})\hspace{1mm}cuenta(x, xs)\hspace{1mm}\leq \hspace{1mm}2}

% Problema 12

\aux{crucesTotales}{xss : [[(\ent, \ent)]]}{[((\ent, \ent), \ent)]}{
concat(\hspace{2mm}[\hspace{1mm}crucesEnMomento(xs)\hspace{3mm}|\hspace{3mm} \selector{xs}{xss}]\hspace{1mm})}

\aux{crucesEnMomento}{xs : [(\ent, \ent)]}{[((\ent, \ent), \ent)]}{
[(x, cuenta(x, xs))\hspace{2mm}|\hspace{2mm}\selector{x}{listaSinRepeticiones(xs)},\newline
cuenta(x, xs) \geq 2]}

\aux{posicionesPorMomento}{ds : [Drone]}{[[(\ent, \ent)]]}{[\hspace{2mm}listaVertical(d)[i]\hspace{3mm}|\hspace{3mm} \selector{i}{\rangoca{0}{|vueloRealizado(ds[0])|}}]}

\aux{listaVertical}{ds : [Drone], i : \ent}{[(\ent, \ent)]}{
[vueloRealizado(d)[i]\hspace{3mm}|\hspace{3mm} \selector{d}{ds}]}

\aux{listaSinRepeticiones}{xs : [T]}{[T]}{[xs[i]\hspace{3mm}|\hspace{3mm}\selector{i}{\rangoca{0}{|xs|}},\hspace{3mm} xs[i] \notin xs\rangoca{0}{i}]}


\subsection{Sistema}
% los aux del tipo sistema

% NOTA: (Jonathan)
% EN LOS PROBLEMAS ESTARIA BUENO REEMPLAZAR LOS "para todo i desde 0 hasta el ancho del campo del sistema"
% CON "para todo i en anchoValido(s)"

\aux{anchoValido}{s : Sistema}{[\ent]}{[i\hspace{1mm}|\hspace{1mm} \selector{i}{\rangoca{0}{dameAncho(campo(s))}}]}

\aux{largoValido}{s : Sistema}{[\ent]}{[j\hspace{1mm}|\hspace{1mm} \selector{j}{\rangoca{0}{dameLargo(campo(s))}}]}
\vspace{2mm}

\aux{listaIds}{ds : [Drone]}{[\ent]}{[id(d)\hspace{1mm}|\hspace{1mm}\selector{d}{ds}]}
\vspace{3mm}

\aux{recibePlaga}{s: Sistema}{[(\ent, \ent)]}{concat[adyacentesValidasCultivos(s, i, j)\hspace{1mm}|\hspace{1mm} \selector{i}{anchoValido(s)}, \newline
\selector{j}{largoValido(s)}, contenido(campo(s)) == Cultivo, \hspace{1mm}estadoDelCultivo(s,i,j) == ConPlaga)]}
\vspace{3mm}

\aux{adyacentesValidasCultivos}{s : Sistema, i,j : \ent}{[(\ent, \ent)]}{[v\hspace{1mm}|\hspace{1mm}\selector{v}{adyacentes(i,j)},\newline
\hspace{1mm}enRango(dimensiones(campo(s)), pmr(v), sgd(v)), contenido(campo(s), pmr(v), sgd(v)) == Cultivo]}
\vspace{3mm}

\aux{adyacentes}{(i,j : \ent)}{[(\ent, \ent)]}{[(i-1,j),(i+1,j),(i,j-1),(i,j+1)]}
\vspace{3mm}

\aux{noCambiaEstadoDeCultivo}{n, v : Sistema}{Bool}{(\forall \selector{i}{\rangoca{0}{dameAncho(v)}}, \forall \selector{j}{\rangoca{0}{dameLargo(v)}} \newline 
\hspace{10cm} contenido(campo(n, i, j)) == Cultivo)
\newline \hspace{10cm} estadoDelCultivo(v, i, j) == estadoDelCultivo(n, i, j)}
\vspace{3mm}

\aux{adyacentesValidasDelGranero}{s : Sistema}{[(\ent, \ent)]}{[\hspace{1mm} x\hspace{1mm}|\hspace{1mm} \selector{x}{adyacentes(prm(posicionGranero(campo(s))), \newline sgd(posicionGranero(campo(s)))),\hspace{1mm} enRango(dimensiones(campo(s)), prm(x), sgd(x) ), noHayDrones(x, s)}]}
\vspace{3mm}


\aux{noHayDrones}{x : (\ent, \ent), s :Sistema}{Bool}{(\forall \selector{d}{enjambreDrones(s)}) \hspace{1mm} posicionActual(d) \neq x}
\vspace{3mm}

\aux{dameCultivos}{s : Sistema}{[Parcela]}{ [\hspace{1mm} (i, j) \hspace{1mm} | \hspace{1mm} \selector{i}{\rangoca{0}{dameAncho(campo(s))}} \hspace{2mm} \selector{j}{\rangoca{0}{dameLargo(campo(s))}},\newline contenido(campo(s), i, j) == Cultivo ]}
\vspace{3mm}


\aux{dameCultivosListos}{s : Sistema}{[Parcela]}{[\hspace{1mm} c \hspace{1mm} | \hspace{1mm} \selector{c}{dameCultivos(s)} , estadoDelCultivo(s, prm(c), sgd(c)) == ListoParaCosechar \hspace{1mm}]}
\vspace{3mm}


\aux{porcentajeCultivosListos}{s : Sistema}{ Float }{ |dameCultivosListos(s)| \hspace{1mm} / \hspace{1mm} |dameCultivos(s)| }
\vspace{3mm}

% Auxiliares problema 23
\aux{cantidadDeMovimientos}{d: Drone, s : Sistema}{\ent}{minimoTres(cantidadFertilizante(d), bateria(d),\newline \hspace{1mm}distanciaHastaObstaculoOMargen(d, s))}
\vspace{3mm}

\aux{minimoTres}{a,b,c : \ent}{\ent}{minimoDos(minimoDos(a, b), c)}

\aux{minimoDos}{a,b : \ent}{\ent}{\IfThenElse{a \leq b}{a}{b}}

\vspace{3mm}

\aux{cantidadFertilizante}{d : Drone}{\ent}{|[x\hspace{1mm}|\hspace{1mm} \selector{x}{productosDisponibles(d)}, x == Fertilizante]|}


\aux{distanciaHastaObstaculoOMargen}{d : Drone, s : Sistema}{\ent}{if(obstaculoEnFilaAIzquierda(d, s))\newline
\hspace{1mm} then(prm(obstaculosDelante(d, s)[|obstaculosDelante|-1]) - prm(posicionActual(d)))\newline
\hspace{1mm} else(prm(posicionActual(d) - 0))}

\vspace{3mm}

%\aux{ultimo elemento} para que sea legible lo anterior

\aux{obstaculoEnFilaAIzquierda}{d : Drone, s : Sistema}{Bool}{(\exists \selector{i}{\rangoca{0}{prm(posicionActual(d))}})\newline
\hspace{1mm} contenido(campo(s), i, sgd(posicionActual(d))) \neq Cultivo}
\vspace{3mm}

\aux{obstaculosDelante}{d : Drone, s : Sistema}{[(\ent, \ent)]}{[(i, sgd(posicionActual(d))\hspace{1mm}|\hspace{1mm} \selector{i}{\rangoca{0}{prm(posicionActual(d))}}, \newline
\hspace{1mm} contenido(campo(s), i, sgd(posicionActual(d))) \neq Cultivo]}
\vspace{3mm}


\aux{posicionAfter}{d : Drone}{(\ent , \ent)}{(prm(posicionActual(d))) - cantidadDeMovimientos(d), sgd(posicionActual(d))}
\vspace{3mm}

% No hay error en el rango.
\aux{vuelosAfter}{viejo, nuevo : Drone}{[(\ent , \ent)]}{vuelosRealizados(viejo) ++ [(i,sgd(posicionActual(nuevo))) \hspace{1mm}| \newline
\hspace{1mm} \selector{i}{\rangoac{prm(posicionActual(viejo))}{prm(posicionActual(nuevo))}}] }
\vspace{3mm}

\aux{bateriaAfter}{d : Drone}{\ent}{bateria(d)- cantidadDeMovimientos(d)}
\vspace{1mm}

\aux{productosAfter}{d : Drone}{[Producto]}{[]}

\aux{anchoRecorrido}{posIzq, posDer: (\ent , \ent)}{[(\ent , \ent)]}{[i \hspace{1mm}|\hspace{1mm} \selector{i}{\rangoca{prm(posIzq)}{prm(posDer)}}]}
\vspace{3mm}

\aux{estadoRecienSembradoOEnCrecimiento}{s: Sistema, i, j : \ent}{Bool}{(estadoCultivo(s, i ,j) == RecienSembrado) \lor (estadoCultivo(s, i, j) == EnCrecimiento)}
\vspace{3mm}

\aux{filasSinDronesVolando}{s : Sistema}{[\ent]}{[j \hspace{1mm}|\hspace{1mm} \selector{j}{\rangoca{0}{dameLargo(campo(s))}}, \newline
\hspace{1mm}\neg(\exists d \in enjambreDrones(s))enVuelo(d) \hspace{1mm}\land \hspace{1mm}sgn(posicionActual(d)) == j] }
\vspace{3mm}


%aux del 24

\aux{sinCensar}{s: Sistema, sis: Sistema, d: Drone}{\bool}{(\exists \selector{x} enjambreDrones(s))\hspace{1mm} id(x) == id(d) \newline
\land contenido(campo(s), prm(posicionActual(x)), sgd(posicionActual(x))) == Cultivo \newline
\land estadoCultivo(sis), prm(posicionActual(x)), sgd(posActual(x)) == NoCensado }
\vspace{3mm}


\aux{tieneAlgunHerbicida}{x: Drone}{\bool}{(\exists \selector{p} productosDisponibles(x))\hspace{1mm} p == Herbicida \lor \newline p == HerbicidaLargoAlcance}
\vspace{3mm}

\aux{tieneAlgunPlaguicida}{x: Drone}{\bool}{(\exists \selector{p} productosDisponibles(x))\hspace{1mm} p == Plaguicida \lor \newline p == PlaguicidaBajoConsumo}
\vspace{3mm}

\aux{tienePlaguicidaBajoConsumo}{x: Drone}{\bool}{(\exists \selector{p} productosDisponibles(x)) \hspace{1mm}p == PlaguicidaBajoConsumo}
\vspace{3mm}


\aux{esHerbicida}{p: Producto}{\bool}{(p == Herbicida)\hspace{1mm} \lor \hspace{1mm} (p==HerbicidaLargoAlcance)}

\aux{esPlaguicida}{p: Producto}{\bool}{(p == Plaguicida) \hspace{1mm} \lor \hspace{1mm} (p == PlaguicidaBajoConsumo)}
\vspace{3mm}

\aux{gastoBateria}{p: Producto}{\ent}{\IfThenElse {(p == Plaguicida)}{10}{5}}
\vspace{3mm}

\aux{posicionesHerbizar}{p: Producto, pos: (\ent,\ent), s: Sistema}{[(\ent,\ent)]}{\IfThenElse {(p == Herbicida)}{([pos])}\newline{adyacentesValidasAHerbizar(pos,s)}}
\vspace{3mm}

\aux{adyacentesValidasAHerbizar}{pos: (\ent , \ent ), s: Sistema}{[(\ent , \ent)]}{[pos] ++ [v\hspace{1mm}|\hspace{1mm} \selector{v} adyacentes(prm(pos),sgd(pos)),\newline enRango(campo(s), prm(pos), sgd(pos)),  (contenido(campo(s), prm(pos), sgd(pos)) == cultivo), \newline (estadoCultivo(s, prm(pos), sgd(pos)) == ConMaleza)}
\vspace{3mm}


\aux{quiereHerbizar}{s : Sistema, x, d : Drone}{Bool}{llegoAMaleza(x, s) \newline \land tieneAlgunHerbicida(d) \land bateria(d) \geq 6}
\vspace{3mm}


\aux{quierePlaguizar}{s : Sistema, x, d : Drone}{Bool}{llegoAPlaga(x, s) \newline \land tieneAlgunPlaguicida(d) \land bateria(d) \geq 11}
\vspace{3mm}


\aux{quierePlaguizarBajoConsumo}{s : Sistema, x, d : Drone}{Bool}{llegoAPlaga(x, s) \newline \land tieneAlgunPlaguicidaBajoConsumo(d) \land 6 \leq  bateria(d) \leq 10}
\vspace{3mm}


\aux{llegoAMaleza}{x: Drone, s: Sistema}{\bool}{\IfThenElse {((contenido(campo(s)),
prm(posicionActual(x)), sgd(posicionActual(x)))\newline == Cultivo)}{(estadoCultivo(s, prm(posicionActual(x)), sgd(posicionActual(x))) == conMaleza)}{\False}}
\vspace{3mm}

\aux{llegoAPlaga}{x: Drone, s: Sistema}{\bool}{\IfThenElse {((contenido(campo(s)),
prm(posicionActual(x)), sgd(posicionActual(x)))\newline == Cultivo)}{(estadoCultivo(s, prm(posicionActual(x)), sgd(posicionActual(x))) == ConPlaga)}{\False}}


\end{document} %Termin�!
